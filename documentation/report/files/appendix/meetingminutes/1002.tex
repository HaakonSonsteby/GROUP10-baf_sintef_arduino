\subsection{Participants:}
Everyone present everyone except Asbjørn who had notified he would be gone.

Meeting started at 14.15

\subsection{Agenda}
Present how we have understood the task to the customer and get feedback on that.
Get more information about what Mr. Babak wants us to do, and how it should be done.

\subsection{Additional Information}
Mr. Farshchian wants us to make a couple of simple prototypes, then a big one that will show that our API support more than one social media. This will include using two different social apps the Android phone, for example Facebook and Shindig.

He also wants the API to support communication from the Arduino to Android, for example temp reading or you can push a button and poke a predefined person.

We dont have to use the Open Social java classes, but we should look into the concepts of it and do something similar.

Open Social concepts:
\begin{enumerate}
\item  User
\item Group
\item  Post
\item Stream; user stream and group stream.
\item Group activity streams
\end{enumerate}

Would be nice if the communication library could support other ways of communication than just Bluetooth. There should also be a way to pair the Android app with the Arduino by using MAC address.


Gui/Settings in the t-shirt app.


\subsection{To do for next meeting}
Think about how the WBS can be translated to stories, then to tasks for each sprint.

\subsection{To do for next meeting with the customer}
To do Monday:
Talk with Simone about the shopping list, find out what is already available for us.
Check out battery drain

\subsection{Notes from Henrik:}
Two simple prototype and one more advanced prototype

Intents library
Class intents
Stream(timeline)
Group
                
GroupStream
Post
User                

Request Intents
Callbacks
Documentation for intents in Android style

Communication Library
C Wrapper
Find and connect module

Two Social apps

Look at battery drain

near field communication?

Marked integration into last years group project

\subsection{Notes from Bjørnar}
Meeting started @ 14.15

Everyone except Asbjorn present. (notified he would be away)

Talk to the customer that we have had some problems understanding what he wants us to do, and that we want to  make sure that we have understood him correctly before we go from this meeting.

Johan and Emanuel explaining on the whiteboard:
\begin{itemize}
\item Social apps, communicate with intents with our API then to the android/arduino app, the android/arduino app will communicate with bluetooth through our com-layer.
\item What objects that we want to use, defined from the open social standard
\item Up for discussion: how strictly we should follow the open social standard
\item Follow the overall concepts that open social is using
\item API layer will provide parsing from intents to intents.
\item The API is a part of the android/arduino app
\item Use the intent services that are already there, try to specialize the intents for our use
\item Someone who made an app store last year, check into it, can upload any files to it. Student project.
\item UBI home app
\end{itemize}

Henrik is asking about how to make requests from for example t-shirt to the social app:
\begin{itemize}
\item In the library you will have specific things you can ask about, the t-shirt can send intents, but if it wants to use our intents, have for example osnap.* infront of it
\item Look up how to register the app for showing up in menu when you want to share anything
\item Talk with Simone about the shopping list
\end{itemize}

Johan explaining about the extended the t-shirt prototype:
\begin{itemize}
\item QR, bluetooth, supporting more than one social network
\item Extra hw devices that we have thought about, vibration, sound, display, leds
\item News feed to sound, settings page for the t-shirt app, for what should be doing what on the t-shirt
\item Disable/enable modules, what goes where.
\end{itemize}

How to go the other way? \newline
Temp sensor for example, post temp to facebook.
  
Think about if you can make the t-shirt app in a visual editor \newline
To custom make your own t-shirt, related to next phase, not priority (qued)

Have to think about, make a list of specifics of what we want to do. \newline
The t-shirt app, would be cool with more than one thing. Make a couple of simple prototypes, and one big one.
Point is to show that we can do more than one thing, proof it's not hard coded.

Anders talking about the bluetooth:
\begin{itemize}
\item How to connect to a specific bt device
\item Every bt have a mac adress, can be send with the QR code.
\end{itemize}

Communication library:
\begin{itemize}
\item Documentation of the intents, how they work etc.
\item Currently have two simple apps, one sending and one recieving.
\item Social intents library, t-shirt app, talk with group 11 about fall detection??
\item Two social apps, social library, communication library, android apps for prototypes, and firmware for the arduino, documentation for the android action intents, make available on the open source market.
\item Communication library, should it be just BT, or do we have to support other wireless communications?
\item Currently it is transparent, should be possible to implement other kinds of wireless comms later.
\item Wifi would be nice, for example hotspot, so the t-shirt can be internet available.
\item Think about it, not critical
\item NFC tags at the lab, so it would be possible to test it out.
\end{itemize}

Misc:
\begin{itemize}
\item Important to be under Apache license. End user contribution agreement.
\item Look into the market, we will have to get client, and how it works with QR codes etc.
\item Shindig, folder with open social concepts implemented as java classes
\item Some kind of management tool, to remove/update etc
\item Customer want the meeting report agenda.
\end{itemize}
