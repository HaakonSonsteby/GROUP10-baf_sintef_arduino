\subsection{Participants}

 Everyone except Jonas

\subsection{Agenda}
Presenting the social Lib and its functions, discuss on further work and figure out should be the next step.

Show temperature prototype working with communication library

\subsection{Additional Information}
Group still waiting for the hardware that was ordered.

Agreement on backwards compatibility of the social library. So that when new social networks comes out, it should be possible to add abstractions specific to a given network without destroying the already existing functionality of the social library. This shouldn't affect the end user of a product, neither the developer. 

We should make the API as simple and user-friendly as possible for the developer. So the developer don't have to use a long time getting into using and understanding the API.

Go with a middle way of how the API is working, don't try to support everything, just the common things that most social networks share. Ie: the concept of user, post(message), stream.
This can include some specific functionality to a given social network. 

Use for example annotations for specifying which social network a given functionality belongs to: FP.post(facebook) / TW.post(twitter). Give the developer possibility to define own annotations as well as using the predefined ones.

\subsection{To do}

\begin{itemize}
\item  Make a target API version 1, make it as simple as possible.
\item Implement classes with simple ideas, that we are sure won't change.
\item Get feedback on the API.
\item Look into using the remoteMethodHandler for Android.
\end{itemize}



\subsection{To do for next meeting with the customer}

\begin{itemize}
\item Use QR code for setting the bluetooth MAC address on the temperature prototype.
\item Get a working prototype to work end to end.
\end{itemize}