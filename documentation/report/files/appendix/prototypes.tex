\newpage
\section{Planning of prototypes}

\subsection{Prototype 1: The Social T-Shirt}

This is our main prototype to show the social library. A t-shirt or sweater connected to a Lilypad Arduino will have several displays and indicator which will all receive data from the Android phone. The data sent will be social media statistics or updates.




\subsection{Prototype 2: Temperature Sensor for Android}

This prototype will primarily be made to show communication going from the Arduino to the Android over our libraries. A temperature sensor will be connected to an Arduino board and then constantly send it's output over bluetooth to a running application on Android. 

\subsection{Prototype 3: Wearable Pictures}

This prototype was first planned to be LED lights showing what mood you had on MySpace. Red LED lights would be angry, green would be happy and so on. We discovered that MySpace had removed this feature from their pages, however the support for it was still in the API available. Yet, it would be a fools errand to pursue something you could not test properly in a real world situation, so we changed the concept of this prototype.
It ended up to be a variant of wearable fashion. One can take a picture or find a picture on your Android phone and choose the "Send" functionality and then choose our program to handle the Send. The Send is actually an intent in Android context, so it is an easy process to start our program from this first intent and bundle with it the picture in question. For prototyping purposes the wearable product will feature just a few LEDs which our Android prototype program will map to certain pixels and then transmit over Bluetooth to the Arduino component and this will promptly light the LEDs. 


%% Pictures here