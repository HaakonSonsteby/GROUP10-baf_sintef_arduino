
\section{Status Reports}


\subsection{Week 4}

PROJECT 10 Open Source Network Arduino Platform(oSNAP) SUMMARY STATUS
REPORT Week 4 1 Introduction First weeks status report where major
directional decisions will be explained. 2 Progress summary The starting
weeks work was mainly focused on identifying the specifics of our
task and for every group member to start individual research into
possibilities in Arduino, Facebook APIs, other social network APIs
and then reporting it back to the group at our meetings. Together
we tried to find good concepts for the Tangible User Interface(TUI)
product that we was going to make. We had a meeting with the customer
on the friday where we discussed our concepts. In this meeting the
customer voiced wishes for the project to go in a more general direction
than focusing on specific tangible end products. We was going to make
a framework where creating products with an Arduino chip was lessened
in complexity both for the developer and the end user. We also got
Arduino products(Bluetooth module etc) from Simone Mora from IDI.
3 Open / closed problems The direction the project should pursue in
regards to the TUI and platform choice became clear on the meeting
with the customer. How well the Arduino chip could handle wireless
connections(Bluetooth, WiFi or Zigby) is an open problem. 4 Planned
work for next period In the meeting with the customer we decided on
a task for the next period. If we could show that the Arduino could
be linked with an Android and/or PC through a wireless connection
we could further refine the requirements and goals of the project.
5 Updated risks analysis Wireless connectivity; if the Arduino chip
addon modules(called SHIELDS) for Bluetooth etc. was too hard to implement
we would have to reconsider wireless connectivity. We set the “get
wireless to work” deadline to be one month. Otherwise we would have
to use only cabled connections and that would be clumsy on a lot of
cool concepts. Likelihood: 4 Impact: 6 Importance: 24


\subsection{Week 5}

PROJECT 10 Open Source Network Arduino Platform(oSNAP) SUMMARY STATUS
REPORT Week 5 1 Introduction A report of a very productive week with
some breakthroughs and setbacks. 2 Progress summary We started of
by working on the tasks set forth. The bluetooth module(SHIELD) we
got to work much faster than we planned for so we set the bar higher
and started working on a general protocol to be used for communication
between Arduino and computers regardless of technology(USB,BT,WiFi
etc). We also had people work on getting a working Facebook application
fetch data from Facebook(last post on Facebook wall). Getting a bluetooth
connection between Android and Arduino was also a challenging task
that we got working by the end of the week. Our results this week
was presented to the customer on the friday as we had agreed on. Our
progress relative to our plan is very good, we are actually ahead
of schedule. We also started working on the preliminary report that
is due in WEEK 6. We chose to write this in LaTeX language as this
is the most powerful way to customise the style of our report. In
the meeting with the customer we had to revise our plan regarding
to the Android framework. The way social networks/applications connect
to our framework should be with intents(an Android standard) and the
framework should be highly flexible in regards to which applications
can work on it. This lead us to reject some of the code we had made
for Facebook fetching, Android Bluetooth talking amongst other things.
We also established working process management tools in GitHub and
ScrumDo(www.scrumdo.com). 3 Open / closed problems Closed: wireless
connectivity to Arduino is a reality and Bluetooth was decided upon
as this is a standard in all mobile smart phones these days so the
possibilties are bigger. Open: The concept of intents on Android needs
to be understood and tried in action. 4 Planned work for next period
Revise established plans on all levels of our project. Set the plans
in motion. 5 Updated risks analysis The risk of wireless connectivity
not working is void after this weeks progress. 
