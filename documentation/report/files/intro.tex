\section{Problem Description}
Social networking sites such as media have normally been accessed  via a web browser. Lately mobile versions have 
become popular, and allow access to context data such as location of users. We foresee a much bigger popularity 
of mobile and pervasive interfaces to social media in the coming years. 

Arduino\cite{link:arduino} is a  platform that allows software to be written in order to construct hardware 
prototypes. In this task the students will develop an Arduino-based interface to Facebook. The information from 
Facebook will be made available on Arduino displays, and user’s interaction with Facebook will be supported by 
Arduino sensors and actuators.

In case Facebook data access shows to be difficult, the students will use Shindig\cite{link:shinding} which 
is an open source implementation of OpenSocial APIs. The service specification is developed in the European project 
SOCIETIES in collaboration with European partners. The students are however expected to innovate and come up 
with new concepts, user interfaces, and service organization

\section{The context}
In the past few years, social networks have become more and more popular, continuously increasing their userbase.
The project task is to design and develop an open source Android framework to allow quick and flexible
development of wireless T.U.I. (Tangible User Interfaces) for social networks using Arduino.
The product will serve both as a proof-of-concept and as starting point for future software projects.

\section{The customer}
Our customer is SINTEF, represented by Mr. Babak A. Farshchian.
Headquartered in Trondheim, Norway, SINTEF is the largest independent research organisation in Scandinavia.
Every year, SINTEF supports research and development at 2,000 or so Norwegian and overseas
companies via its research and development activity.

\section{The Team}
The team consisted of seven students from NTNU. About half of the members have worked
together on previous projects and thus shared some teamwork skills and experiences.
Having a properly functioning team where all the skills of each team member are properly
utilized is vital for the success of any project. None of the team members had 
worked for a customer before.

\subsection{Team members}

\begin{itemize}
\item{\anders}\newline
Third year Informatics student at NTNU. Experience with the programming languages Java,
C and C++. Also experience with Arduinos, AVRs and some basic electronics knowledge.

\item{\henrik}\newline
Third year student at NTNU. Main language Java, basic knowledge in 3D and electronic circuits.

\item{\johan}\newline
Third year student at NTNU. Experience with the programming languages Java, C, C++  and
Basic. Worked with AVR microcontrollers in other projects and courses.

\item{\asbjorn}\newline
Second year student on bachelor in computer science at NTNU. Has previously had one year of
introductory psychology which is to be included in the bachelor in computer science.
Experience with Java.

\item{\emanuele}\newline
Third year Computer Science exchange student from Tor Vergata University, Rome.
Experience with Java, C and C++ programming.

\item{\jonas}\newline
Third year student at NTNU. Experience with Java, Python, Lua, PHP, and web standards such as HTML,
CSS and Javascript. Previously worked with media and sound engineering.

\item{\bjornar}\newline
Third year Informatics student at NTNU. Experince with programming in Java, C and C++.
Previously worked with electronics and have basic knowledge about AVR microcontrollers through
own projects.
\end{itemize}
