\section{Process management}


With our weekly or biweekly meetings with the customer in mind as well as his request for intermediate prototypes to be presented at these meetings, the choice of scrum with it’s sprints that last as long as these intervals is natural. That as well as everyone on the group having previous experience with the scrum process through other projects.

Modifications to our project


To fit with our schedule we will modify the scrum model somewhat. We will move the daily meetings to be three times a week(Monday,Wednesday and Friday) as well as doing work sessions on these days. Sprints will also be modified to primarily be biweekly and may vary between sprints according to our meetings with the customer and as the intermediate prototypes we will present to him are finished. Sprint backlogs will be very crucial for our use. With the abstract goals of the project set forth, the gradual inclusion of new features as we and the customer see the project evolve will be mirrored in the backlogs from sprint to sprint. We will also maintain a product backlog with input from the customer. To properly fit our product goals the differences between faculties of a normal product backlog and the requirements and wishes of the customer we will have to modify the backlog to be somewhat of an intersection between requirements and feature goals. Another important difference to normal scrum evolution will have to be the “definitions of done”. As our end state is diffuse in regards to a tangible product we have in collaboration with the customer some design goals(see requirements) that will be the main focus of our project.




