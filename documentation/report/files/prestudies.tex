\section{Communication resources}

The team exchanged both e-mails and mobile numbers. A permanent Skype
group chat on which we meet on a daily basis was set up, a mailing
list was also created. Several documents of interest to the group
were made available to everyone using Google Documents.


\section{Time resources}

A meeting table was arranged during the first meeting. Team members
have weekly meetings on Monday, Wednesday and Friday at twelve o'clock.
Meetings with the client were arranged for each Friday at two fifteen.


\section{Existing solutions}

We have done some research on the internet about similar and related products. We found an interesting
Facebook like indicator that would glow when some content the user posted on Facebook was liked by someone.
It was realized with an Arduino board inside a giant Lego hand. Another funny product is a US-mailbox rising
its 'arm'  whenever the user received a notification on Facebook. The relevancy of these existing products
for us was to confirm that it was possible to connect the user with Facebook in a tangible way using an Arduino.
Our own ideas do not have to differ much from already existing concepts, our goal is in fact to show that we can
develop a general concept that would encompass these ideas if that was the case. We also looked into social
networks other than Facebook like Twitter or LinkedIn. Facebook is definitely the most popular social network
out there, but it uses a proprietary API that is not compatible with the other social networks's or open
API implementations.

Ultimately we presented various ideas and concepts to the customer and let him decide on what he would
like the group to work on. As a prototype he would like to see a T-Shirt with a built-in small LCD screen
that will display user's data fetched from social networks, Facebook and Twitter.


\section{Software and hardware development tools}

Various software tools will be used througout the project. These include:
IDEs of choice: NetBeans and Eclipse
Collabaration tools: git, svn, Trac
Pre-existing software: Facebook Android SDK,    Android SDK, Twitter4j, Scribe and possibily others

The project task requires various hardware, namely Arduino boards, shields and modules and Android mobiles. Our customer kindly arragend for us so that we can borrow Arduino equipment from the lab. The team will use its own
mobiles for running, testing and prototyping the system.


\section{Add more stuff}

Something something