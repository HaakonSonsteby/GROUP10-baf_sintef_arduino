
This chapter describes both the functional and non-functional requirements
for the various parts of the project.

\section{Functional requirements}

\begin{table}
		\caption{Requirements}
\begin{tabular}{ | l | c | }
	\hline                        
	\bf{Requirement} & \bf{Priority} \\
	\hline
	F1:  SocialLib					& Very High \\
	F2:  ComLib						& Very High \\
	F7:  T-Shirt Prototype			& High \\
	F4:  T-Shirt application		& Medium \\
	F3:  oSNAP application			& Medium \\
	F5:  Facebook application		& Medium \\
	F6:  Twitter application		& Medium \\
	F8:  Temperature application 	& Low \\
	F10: LED matrix application		& Low \\
	F9:  Temperature Prototype		& Low \\
	F11: LED matrix Prototype		& Low \\
	\hline  
\end{tabular}
\end{table}

\subsection{F1: Social library}
\begin{description}
	\item[Network support:] The library shall support at least two social
	networks. One shall be Facebook.
	\item[Abstractions:] The library shall provide abstractions for concepts
	found in social networks, and simplify Android's IPC mechanisms
	between social services and TUI prototypes applications.
\end{description}
	
\subsection{F2: Communication library}
\begin{description}
	\item[Wireless connectivity:] Since the product will target a user-base
	with no technical background, the customer required all connections
	between devices to be wireless, so that connecting the User Interface
	to the Android mobile will be as easy as possible for the end-users.
	For this reason, the technical details of the connection should be
	hidden to the end-user so that the product can be easily operated.
	\item[Two-way communication:] The communication shall be two-way:
	from the Arduino device to Android and vice-versa.
\end{description}

\newpage

\subsection{F3: oSNAP application}
\begin{description}
	\item[Libraries:] The application shall use the Communication library
	to communicate with TUI prototypes.
\end{description}

\subsection{F4: T-Shirt application}
\begin{description}
	\item[On and Off:] The user shall be able to turn the application On
	and Off. When the application is not running, no social data will be
	forwarded to the T-Shirt prototype.
	\item[Libraries:] The application shall use the Communication library to
	communicate with the T-Shirt prototype and the Social library to
	receive and send data from/to social services.
	\item[Rules:] The application shall let the user setup a set of rules
	to control the behavior of the T-Shirt. Rules can be created and deleted.
	\item[User interaction:] The application shall continue to work
	when the mobile itself is idle.
\end{description}

\subsection{F5: Facebook application}
\begin{description}
	\item[Login (Authentication):] The application shall handle the
	authentication with Facebook.
	\item[Logoff:] The user shall be able to log off once logged in.
	\item[Facebook connectivity:] The application shall fetch and push data
	from/to Facebook as requested, even when the mobile itself is idle
	(not operated).
	\item[Libraries:] The application shall use the Social library to
	communicate with TUI prototype applications.
	\item[User input:] Once the user is authenticated, no further user action
	should be required.
\end{description}

\subsection{F6: Twitter application}
\begin{description}
	\item[Libraries:] The application shall use the Social library to
	communicate with TUI prototype applications.
\end{description}

\subsection{F7: T-Shirt prototype}
\begin{description}
	\item[Features:] The prototype shall map some social content to various
	embedded electronic devices.
	\item[Libraries:] The prototype shall use both the Social and Communication
	libraries.
	\item[User input:] Once the rules for the T-Shirt are set, no further user
	input shall be required.
\end{description}

\subsection{F8: Temperature prototype}
\begin{description}
	\item[Features:] The prototype shall read the temperature using an Arduino
	sensor and send it to Facebook.
	\item[Libraries:] The prototype shall use both the Social and Communication
	libraries.
\end{description}
	
	\subsection{F9: LED matrix prototype}
	\begin{description}
		\item[Features:] The prototype consists of a matrix of LEDs controlled by an application on 
		an Android smart phone.
		\item[Libraries:] The prototype shall use the Communication library.
		\item[]
	\end{description}

\subsection{F10: Temperature application}
\begin{description}
	\item[Mac address:] The user should be able to scan the mac address from a QR code, and/or enter it manually. 
	\item[Share temperature:] The user should have full control over when and to which social media the temperature data is shared.
	\item[Time interval:] The user should be able to set automated updating of the temperature.
	\item[Dependecies:] Facebook service application.
	\item[Libraries:] The application shall use the Social and Communication libraries.
\end{description}

\subsection{F11: LED matrix application}
\begin{description}
	\item[?] ?
\end{description}

\newpage

\section{Non functional requirements}

\subsection{Libraries}

Both libraries share some non functional requirements.
For the sake of simplicity they will be listed together.

\begin{description}
	\item[Extensibility:] The library shall have a modular design
	in order to be easily extended. The software that will be developed for
	this project will serve as a proof-of-concept and possibly as a starting
	point for other research projects. For this reason a flexible, modular
	software architecture is an important requirement for our customer. The code
	shall be developed in independent, thus reusable modules so that adding new
	functionality will be fairly easy for new developers.
	\item[Target platform:] The library shall be compatible with the Android
	system starting from version 2.2. This will place restrictions on the
	existing software and on the programming languages that will be adopted.
	\item[Documentation:] The library shall be documented using Javadoc.
	Since the product is going to be further developed by other people and is
	mainly designed to simplify application development, a good documentation is
	important. Proper documentation and tutorials should be available to
	third-party developers.
	\item[Licensing:] The library shall be released under the Apache License 2.0
	\footnote{See http://www.apache.org/licenses/LICENSE-2.0.html for more
	information on the Apache 2.0 license}. The customer made clear that all the
	software developed needs to be released under a permissive, Apache software
	license. This implies that pre-existing software that will eventually be
	adopted and incorporated in the project must have a compatible license.
\end{description}


\subsection{Prototypes}

Non-functional prototypes' requirements.

\subsection{Applications}

All the applications share some non functional requirements.
For the sake of simplicity they will be listed below:

\begin{description}
	\item[Target platform:] The applications shall be compatible with the
	Android system starting from version 2.2. This will place restrictions on
	the software and on the programming languages that will be adopted.
	\item[Documentation:] The applications shall be documented using Javadoc.
	\item[Licensing:] The applications and their source code shall be released under the Apache
	License 2.0.
\end{description}
