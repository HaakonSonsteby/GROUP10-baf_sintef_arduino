
\section{Preliminary requirements}
\begin{itemize}
\item Wireless connectivity
\end{itemize}
Since the product will target a user-base with no technical background,
the customer required all connections between devices to be wireless,
so that connecting the User Interface to the Android mobile will be
as easy as possible for the end-users. For this reason, the technical
details of the connection should be hidden to the end-user so that
the product can be easily operated.
\begin{itemize}
\item Flexible software architecture
\end{itemize}
The software that will be developed for this project will serve as
a proof-of-concept and possibily as a starting point for other research
projects. For this reason a flexible, modular software architecture
is an important requirement for our customer. The code shall be developed
in independent, thus reusable modules so that adding new functionality
will be fairly easy for new developers.
\begin{itemize}
\item Software licenses
\end{itemize}
The customer made clear that the software developed needs to be released
under a permissive, Apache compatible software license. This implies
that pre-existing software that will eventually be adopted and incorporated
in the project must have a compatible license.
\begin{itemize}
\item Working prototype
\end{itemize}
To show that the concept of Tangible User Interfaces for Android applications
(including social networks) using Arduino is not only possible, but
also a feasible market product, the customer is interested in a working
prototype. As the project evolves, together with our customer we will
gradually decide on the specifications of the prototype that will
be produced at the end of the project. Such prototype will play an
important role for the customer satisfaction.
