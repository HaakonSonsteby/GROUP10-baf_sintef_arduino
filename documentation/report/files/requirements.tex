
This chapter describes both the functional and non-functional requirements
for the various parts of the project.

\section{Functional requirements}

\begin{table}
		\caption{Requirements}
\begin{tabular}{ | l | c | }
	\hline                        
	\bf{Requirement} & \bf{Priority} \\
	\hline
	F2: ComLib & Very High \\
	F1: SocialLib & Very High \\
	F5: T-Shirt Prototype & High \\
	F4: T-Shirt App & Medium \\
	F3: Facebook App & Medium \\
	F6: Temperature Prototype & Low \\
	F7: LED Matrix Prototype & Low \\
	\hline  
\end{tabular}
\end{table}

	\subsection{F1: Social library}
	\begin{description}
		\item[Network support:] The library shall support at least two social networks.
		One is OpenSocial and the other is Facebook.
		\item[Abstractions:] The library shall provide abstractions for concepts found
		in social networks.
	\end{description}
	
	\subsection{F2: Communication library}
	\begin{description}
		\item[Wireless connectivity:] Since the product will target a user-base with
		no technical background, the customer required all connections between
		devices to be wireless, so that connecting the User Interface to the Android
		mobile will be as easy as possible for the end-users. For this reason, the technical
		details of the connection should be hidden to the end-user so that the product can
		be easily operated.
		\item[Two-way communction:] The communication shall be two-way: from the
		Arduino device to Android and vice-versa.
	\end{description}

	\subsection{F3: Facebook application}

	\begin{description}
		\item[Login (Authentication):] The application shall handle the authentication with Facebook.
		\item[Logoff:] The user shall be able to log off once logged in.
		\item[Facebook connectivity:] The application shall fetch and push data
		from/to Facebook as requested, even when the mobile itself is idle (not operated).
		\item[Libraries:] The application shall use the Social library to send and receive
		data to/from the T-Shirt application
		\item[User input:] Once the user is authenticated, no further user action is required.
	\end{description}

	\subsection{F4: T-Shirt application}
	
	\begin{description}
		\item[On and Off:] The user shall be able to turn the application On and Off.
		When the application is not running, no social data will be forwarded
		\item[Libraries:] The application shall use the Communication Library to
		communicate with the T-Shirt prototype and the Social library to
		receive and send data from/to the Social Application.
		\item[Rules:] The application shall let the user setup a set of rules
		to control the behavior of the T-Shirt. Rules can be created and deleted.
		\item[User interaction:] The application shall continue to work
		when the mobile itself is idle.
	\end{description}

	

	\subsection{F5: T-Shirt prototype}

	\begin{description}
		\item[Features:] The prototype shall map some social content
		to various embedded electronic devices.
		\item[Libraries:] The prototype shall use both the Social and
		Communication libraries.
		\item[User input:] Once the rules for the T-Shirt are set,
		no further user input shall be required.
	\end{description}

	\subsection{F6: Temperature prototype}
	\begin{description}
		\item[Features:] The prototype shall read the temperature
		using an Arduino sensor and push it to Facebook.
		\item[Libraries:] The prototype shall use both the Social and
		Communication libraries.
		\item[]
	\end{description}
	
	\subsection{F7: Image to LED matrix prototype}
	\begin{description}
		\item[Features:] The prototype shall display an image sent by a Android
		mobile using some LEDs.
		\item[Libraries:] The prototype shall use the Communication library.
		\item[]
	\end{description}



\section{Non functional requirements}

\subsection{Libraries}

Both libraries share some non functional requirements.
For the sake of simplicity they will be listed together.

\begin{description}
	\item[Extensibility:] The library shall have a modular design
	in order to be easily extended. The software that will be developed for
	this project will serve as a proof-of-concept and possibily as a starting point
	for other research projects. For this reason a flexible, modular software
	architecture is an important requirement for our customer. The code shall be
	developed in independent, thus reusable modules so that adding new functionality
	will be fairly easy for new developers.
	\item[Target platform:] The library shall be compatible with the Android
	system starting from version 2.2. This will place restrictions on the software
	and on the programming languages that will be adopted.
	\item[Documentation:] The library shall be documented using Javadoc.
	Since the product is going to be further developed by other people and is
	mainly designed for easier application development then proper research and usage
	documentation is important. Proper documentation or tutorials should describe how
	the end-user can easily use the libraries without any technical knowledge of how
	the library works.
	\item[Licensing:] The library shall be released under the Apache License 2.0 \footnote{See 
           http://www.apache.org/licenses/LICENSE-2.0.html for more information on the Apache 2.0 license}.
	The customer made clear that all the software developed needs to be released
	under a permissive, Apache software license. This implies that pre-existing
	software that will eventually be adopted and incorporated in the project must
	have a compatible license.
\end{description}


\subsection{Prototypes}

Non-functional prototypes' requirements.

\begin{description}
	\item[Arduino Platform:] Each of the prototypes should be implemented using Arduino\cite{link:arduino} hardware.
	\item[Wireless Conectivity:] The prototypes should show wireless connection signal to be able to
	connect to remote devices.
	\item[Power:] Each of the prototypes need to have some sort of battery power-source to make it
	mobile. There should be no need to plug the prototype device into a wall socket to be able to use it.
\end{description}

\subsection{Applications}

All the applications share some non functional requirements.
For the sake of simplicity they will be listed below:

\begin{description}
	\item[Target platform:] The applications shall be compatible with the Android
	system starting from version 2.2. This will place restrictions on the software
	and on the programming languages that will be adopted.
	\item[Documentation:] The applications shall be documented using Javadoc.
	\item[Licensing:] The applications shall be released under the Apache License 2.0.
\end{description}


