
Testing is an integral part of every development process.
Many different testing approaches have been proposed during the history of
software development, but they all suggest that testing has to be performed continuously
during the development process. This is of particular importance to our process since
it follows an iterative approach. Testing should be planned in the early stages of each
iteration for each testable unit that is going to be implemented and performed as soon as
the iteration is complete. As the system evolves so does the testing. It should be more comprehensive and cover each module and the
system itself as a whole.

\section{Customer feedback}
Customer feedback was an important part of the testing approach for our project.
Working prototypes and/or documentation were produced for each meeting with the customer
in order to test the features implemented in the prototypes, receive a general acceptance
on our iteration goals and plan the next iteration. Receiving a constant feedback from the
user was very important as especially the 'social' part of the project didn't really have a
specific set of requirements from the beginning, so acquiring feedback from the customer allowed
us to understand the most common use case scenarios and from those obtain a set of requirements used
to design the system.

\section{Unit testing}
This involves testing small portions, like methods and functions, of the code, making sure they work 
as intended throughout the process of implementing code. Tests will be executed after changes in the 
code to make sure that it still works as expected, this means that unit tests have to be written for 
large portions of the API.

\section{User interface testing}
For the prototypes we made some applications that will run on an Android phone. To make sure our 
prototypes are understandable and easy to use, we have used group members that have not been 
involved in the process of making the UI, as test candidates. It might not be the ideal way of testing 
an UI, but in the given timeframe of this project it was the easiest and fastest way of doing it.

\section{Integration testing}
After each module of the API has been Unit tested, they will be put together to form bigger components 
of the working system. This is to make sure that the smaller modules will work correctly when placed 
into bigger components. 

In our case this will be to run tests on ComLib and SocialLib making sure they work individually, 
before they are tested together.

\section{System testing}
This involves butting the components from integration testing together to form a complete system that 
can be tested. Preferably each module will be added incrementally, to easier spot if any modules produce errors.