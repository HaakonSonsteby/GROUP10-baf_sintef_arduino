
Testing is an integral part of every development process.
In iterative software development processes, testing is usually planned in the
early stages of each iteration and planned as soon as the iteration itself is
complete. As the system evolves so does the testing; it should be more
comprehensive and cover not only modules but also the system itself as a whole.

\section{Testing approach}
Testing can be performed in a top-down or bottom-up fashion.

The top-down approach consists in integrating modules incrementally, moving
downward in the hierarchy in either a breadth-first or depth-first fashion. \newline
Subordinate dummy modules (stubs) are substituted one at time with the real ones
and tested. Regression testing follows each integration increment.

On the other side the bottom-up approach begins testing atomic modules,
sometimes combining them into clusters, and moves upward in the program
structure. Sometimes 'driver' programs are produced in order to test \newline
non-standalone components like libraries. This approach eliminates the need for
dummy modules, but the system as whole doesn't exist as an "entity" until the
last module is integrated.

In our case, since we presented many prototypes to the customer, we needed to
have an implementation of the system as a whole quite early. We coded various
mockup applications that acted as 'drivers' in order to demonstrate and test the
functionalities of the libraries. Stubs (or dummy modules) were also coded
to supply functionalities not yet fully implemented.

Some of these 'driver' programs were included in the source coded repository.

\todo {
	mention that mockups have their own package: no.ntnu.osnap.mockups
	add a link to package structure maybe?
}

\section{Customer feedback}
Receiving a constant feedback from the customer was very important as the
project didn't really have a specific set of requirements from the beginning.
Many prototypes were presented during meetings with the customer in order to
test the features implemented so far, receive a general acceptance on our
iteration goals and plan the next iteration. Feedback from the customer
allowed us to understand the most common use case scenarios and from those
obtain a set of requirements used to design the system and plan test cases.
Customer's feedback and proved to be very valuable for the project.

\section{Testing strategy}

\subsection{Unit testing}
Unit testing involves testing small portions of code or of the system itself
(modules). Where possible, functions have been tested indipendetly. In order to
test non indipendent system modules, such as the libraries, mockup (driver)
applications were coded.

\subsection{Integration testing}
Once the system modules were completed they were integrated one by one into the
final system. After each stub/driver had been swapped with its correspective
module, it was tested again to ensure it was working properly.

\subsection{System testing}
System testing constists in testing the system as a whole.
In our case this consisted in testing our final prototype, performing a series
of tests involving all the parts of the system from the Android applications
to the hardware components.

\subsection{User interface testing}
Since our product encompassed various Android applications, we performed some
user interface testing. To make sure the user interfaces are understandable and
easy to use, we have delegated their testing to group members that have not been
involved in their design and coding.

\section{Functional requirement testing}
Since the project a combination of proof-of-concept and research project,
tests are planned in order to test the functionality of the various parts of the
system against their own requirements and not for particular scenarios or more
specific use cases.

Tests were performed in order: the hardware and libraries were tested first
(the Android applications depended on both) and then integrated into more
complete tests involving all the parts of the T.U.I. prototype: Android
applications, libraries and the Arduino hardware.


\begin{table}[h!]
\begin{tabular}{|l|p{10cm}|}
\hline Test ID: &		[F1] Social lib. / [F5] Facebook app. / [F6] Twitter app.\\
\hline Purpose: &		Testing the Social library and social applications
						functionality \\
\hline Precondition: &	A mockup TUI prototype application is installed on an
						Android mobile, as well as the FB and Twitter apps. \\
\hline
Steps:
 & 1: Start the Facebook application \\
 & 2: Press 'Login' button \\
 & 3: Enter user's credentials and proceed with log in \\
 & 4: Exit the Facebook application \\
 & 5: Start the Twitter application \\
 & 6: Exit the Twitter application \\
 & 7: Start the mockup prototype application \\
 & 8: Press 'Discover social services' button \\
 & 9: Press 'Make a request' button \\
\hline
Expected result:
 & User is correctly authenticated (Facebook) \\
 & The Facebook social service is discovered by the mockup app \\
 & The Twitter social service is discovered by the mockup app \\
 & The mockup app. receives the correct response based on the request made by
	both social networks \\
\hline
Actual Result:
 & Passed \\
\hline
\end{tabular}
\caption{Test F1/F5/F6}
\end{table}


\begin{table}[h!]
\begin{tabular}{|l|p{10cm}|}
\hline Test ID: &		[F7] T-shirt prototype \\
\hline Purpose: &		Test T-shirt prototype functionality \\
\hline Preconditions: & The Arduino is connected to a PC via USB, all components
						are connected to the Arduino board; the Arduino board is
						on.\\
\hline
Steps:
 & 1: Toggle a LED status \\
 & 2: Send a signal to the vibrator module \\
 & 3: Print some text to LCD display \\
 & 4: Send sound to speaker \\
\hline
Expected result:
 & LED is toggled on/off \\
 & Vibrator works \\
 & Correct text is displayed on LCD \\
 & A sound is emitted by the speaker \\
\hline
Actual Result:
 & Passed \\
\hline
\end{tabular}
\caption{Test F7}
\label{tbl:f7test}
\end{table}


\begin{table}[h!]
\begin{tabular}{|l|p{10cm}|}
\hline Test ID: &		[F9] Temperature prototype \\
\hline Purpose: &		Test temperature prototype functionality \\
\hline Precondition: &	The Arduino board is connected to a PC via USB and
						turned on; the temperature sensor and a LCD screen are
						connected to the Arduino \\
\hline
Steps:
 & 1: Press a hardware button \\
\hline
Expected result:
 & Temperature is displayed on Arduino's LCD \\
\hline
Actual Result:
 & Passed \\
\hline
\end{tabular}
\caption{Test F9}
\label{tbl:f9test}
\end{table}


\begin{table}[h!]
\begin{tabular}{|l|p{10cm}|}
\hline Test ID: &		[F11] LED matrix. \\
\hline Purpose: &		Test LED matrix prototype functionality \\
\hline Precondition: &	The Arduino is connected to a PC via USB, and turned on;
						the LED matrix is connected to the Arduino. \\
\hline
Steps:
 & 1: Send sample data to Arduino \\
\hline
Expected result:
 & Correct LED light up according to sample data \\
\hline
Actual Result:
 & Passed \\
\hline
\end{tabular}
\caption{Test F11}
\label{tbl:f11test}
\end{table}


\begin{table}[h!]
\begin{tabular}{|l|p{10cm}|}
\hline Test ID: &		[F11] LED matrix prototype / [F10] LED matrix app. /
						[F2] Comm. library \\
\hline Purpose: &		Testing the LED matrix prototype and application \\
\hline Precondition: &	LED matrix applications is installed on the mobile and
						the Arduino board is powered \\
\hline
Steps:
 & 1: Pair cell phone with Arduino board using Android OS \\
 & 2: Start LED matrix application \\
 & 3:  \\
 & 4: Press '' button \\
\hline
Expected result:
 & 1: Correct LEDs are lighted up \\
\hline
Actual Result: &  result \\
\hline
\end{tabular}
\caption{Test F11/F10/F2}
\label{tbl:f11f10f2test}
\end{table}


\begin{table}[h!]
\begin{tabular}{|l|p{10cm}|}
\hline Test ID: &		[F8] Temperature app. / [F1] Social Lib / [F2] Comm. lib \\
\hline Purpose: &		Test the Temperature prototype and application
						functionality  \\
\hline Precondition: &	The two apps are installed and Arduino board is powered
						and paired. \\
\hline
Steps:
  & 1: Start Facebook application \\
  & 2: Press 'Log in' button \\
  & 3: Exit the Facebook application \\
  & 3: Start prototype's application \\
  & 4: Tap the screen to read the temperature\\
  & 5: Press button on the Arduino board to send temperature \\
\hline
Expected result:
  & Temperature is corretly read by the prototype and posted on the logged-in
	user's Facebook wall \\
\hline
Actual Result:
  & Passed \\
\hline
\end{tabular}
\caption{Test F8/F1/F2}
\label{tbl:f8f1f2test}
\end{table}


\begin{table}[h!]
\begin{tabular}{|l|p{10cm}|}
\hline Test ID: &		[F3] oSNAP app. / [F2] Comm. lib \\
\hline Purpose: &		Test connecting to the T-shirt TUI prototype using the
						Comm. lib and oSNAP application \\
\hline Precondition: &	The oSNAP app is installed on an Android phone; the
						Arduino board is installed in the T-shirt and powered. \\
\hline
Steps:
 & 1: Start oSNAP application \\
 & 2: Press 'Scan QR code' button \\
 & 3: Scan QR code on the T-Shirt \\
 & 4: Press 'Toggle LED' button (turn on) \\
 & 5: Press 'Toggle LED' button (turn off) \\
 & 6: Press 'Play sound' button \\
 & 7: Press 'Print "Hello world" button' \\
 & 8: Press 'Vibrate' button \\
\hline
Expected result:
 & The oSNAP application connects to the Arduino \\
 & Led is toggle on and then off \\
 & A sound is emitted \\
 & The vibration module vibrates \\
 & The 'Hello world' string is printed on the LCD \\
\hline
Actual Result:
 & Passed \\
\hline
\end{tabular}
\caption{Test F3/F2}
\end{table}


\begin{table}[h!]
\begin{tabular}{|l|p{10cm}|}
\hline Test ID: &		[F1] Social lib. / [F2] Comm. lib. / [F4] T-shirt app. /
						[F5] Facebook App / [F7] T-shirt TUI prototype \\
\hline Purpose: &		Test T-shirt prototype functionality \\
\hline Precondition: &	The two applications are installed and the T-shirt T.U.I.
						is on and paired with cell phone \\
\hline
Steps:
  & 1: Start Facebook application \\
  & 2: Press 'Log in' button \\
  & 3: Exit the Facebook application \\
  & 4: Start T-shirt app \\
  & 5: Press 'Discover services' button \\
  & 6: Select 'Facebook' from the list of available social services \\
  & 7: Set up rules to define the behavior of the tshirt \\
  & 8: Press 'Connect to Arduino' button \\
  & 9: Press 'Enable service' button \\
  &10: Exit the T-shirt application \\
\hline
Expected result:
   & Depending on the rules that have been set, the correct text is displayed,
   or the correct LED lights up.. and so on \\
\hline
Actual Result:
  &  Passed \\
\hline
\end{tabular}
\caption{Test F1/F2/F4/F5/F7}
\label{tbl:f1f2f4f5f7test}
\end{table}

