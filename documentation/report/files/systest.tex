
Testing is an integral part of every development process.
Many different testing approaches have been proposed during the history of
software development, but they all suggest that testing has to be performed continuously
during the development process. This is of particular importance to our process since
it follows an iterative approach. Testing should be planned in the early stages of each
iteration for each testable unit that is going to be implemented and performed as soon as
the iteration is complete. As the system evolves so does the testing approach which should
then be more comprehensive and cover the integration of each module into the system and the
system itself as a whole.

\section{Customer feedback}
Customer feedback was an important part of the testing approach for our project.
Working prototypes and/or documentation were produced for each meeting with the customer
in order to test the features implemented in the prototypes, receive a general acceptance
on our iteration goals and plan the next iteration. Receiving a constant feedback from the
user was very important as especially the 'social' part of the project didn't really have a
specific set of requirements from the beginning, so acquiring feedback from the customer allowed
us to understand the most common use case scenarios and from those obtain a set of requirements used
to design the system.

\section{Unit testing}

\section{User interface testing}

\section{Integration testing}

\section{System testing}
