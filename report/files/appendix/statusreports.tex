\section{Week 4}

PROJECT 10 open Social Network Arduino Platform (oSNAP)

SUMMARY STATUS REPORT: Week 4

\begin{itemize}
\item Introduction
\end{itemize}
First weeks status report where major directional decisions will be explained.

\begin{itemize}
\item Progress summary
\end{itemize}
The starting weeks work was mainly focused on identifying the specifics of our task and for
every group member to start individual research into possibilities in Arduino, Facebook APIs,
other social network APIs and then reporting it back to the group at our meetings. Together we
tried to find good concepts for the Tangible User Interface (TUI) product that we are going to
develop. We had a meeting with the customer on friday where we discussed our ideas. In
this meeting the customer expressed his requirements for the project to go in a more
general direction than focusing on specific, tangible end product.
We are going to design and develop a framework that will allow developers to create
Arduino related products with ease. Mr. Babak kindly arranged for us so that we could
borrow Arduino products (boards, bluetooth modules etc) from Mr. Simone Mora.

\begin{itemize}
\item Open / closed problems
\end{itemize}
Closed: The direction the project should pursue regarding TUI and platform choice became clearer
on the meeting with the customer.
\\
Open: How well the Arduino device can handle wireless connections
(Bluetooth, WiFi, Xbee) is an open problem.

\newpage
\begin{itemize}
\item Planned work for next period
\end{itemize}
In the meeting with the customer we agreed on a goal for the upcoming week.
If we could show that the Arduino can be linked with an Android and/or PC
through a wireless connection we could further refine the requirements
and goals of the prototype to be produced.

\begin{itemize}
\item 5 Updated risks analysis
\end{itemize}
Wireless connectivity: if the Arduino addon modules (called 'shields') for Bluetooth or other
wireless protocols are too hard to implement we would have to reconsider wireless connectivity.
We gave ourselves a deadline of one month to get wireless working, otherwise we will have to
fallback to cable connections and that will definitely not make our customer satisfied.

Likelihood: L, Impact: H, Overall Importance: M

\section{Week 5}

PROJECT 10 open Social Network Arduino Platform (oSNAP)

SUMMARY STATUS REPORT : Week 5

\begin{itemize}
\item Introduction
\end{itemize}
A report of a very productive week with some breakthroughs and setbacks.

\begin{itemize}
\item Progress summary
\end{itemize}
We started working on the tasks set forth. We got the bluetooth module (or shield) 
working much faster than we planned and so we started working on a general protocol
to be used for communications between Arduino and computers regardless 
of the technology (USB, BT, WiFi and so on). We also managed to read and publish
data from an Android application using the Facebook-Android SDK. Getting a bluetooth
connection between Android and Arduino was also a challenging task that we managed
to accomplish by the end of the week. Our results this week were presented to the customer
on friday as we agreed. Our progress relative to our plan is good, we are actually ahead of
schedule. We also started working on the preliminary report that is due in WEEK 6.
We chose to write this using LaTeX because it's very powerful and professional.
After meeting with the customer on friday we reviewed our plan regarding the Android framework.
The way social networks/applications connect to our framework should involve
Intents (an Android standard). Some of the code and documentation we produced until
now had to be scrapped for this reason. We also established process management and collaborative
tools, namely GitHub and ScrumDo (www.scrumdo.com).

\newpage
\begin{itemize}
\item Open / closed problems
\end{itemize}
Closed: Wireless connectivity to Arduino is a reality. It relies on Bluetooth,
which is featured on almost every mobile.
\\
Open: The concept of intents on Android needs to be understood and tried in action.

\begin{itemize}
\item Planned work for next period
\end{itemize}
Review system architecture on all levels of our project.
Document on the Android Intents mechanism.
Set the plans in motion.

\begin{itemize}
\item Updated risks analysis
\end{itemize}
The risk of wireless connectivity not working is void after this week's progress.